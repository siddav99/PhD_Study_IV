% Options for packages loaded elsewhere
\PassOptionsToPackage{unicode}{hyperref}
\PassOptionsToPackage{hyphens}{url}
%
\documentclass[
]{article}
\usepackage{amsmath,amssymb}
\usepackage{lmodern}
\usepackage{ifxetex,ifluatex}
\ifnum 0\ifxetex 1\fi\ifluatex 1\fi=0 % if pdftex
  \usepackage[T1]{fontenc}
  \usepackage[utf8]{inputenc}
  \usepackage{textcomp} % provide euro and other symbols
\else % if luatex or xetex
  \usepackage{unicode-math}
  \defaultfontfeatures{Scale=MatchLowercase}
  \defaultfontfeatures[\rmfamily]{Ligatures=TeX,Scale=1}
\fi
% Use upquote if available, for straight quotes in verbatim environments
\IfFileExists{upquote.sty}{\usepackage{upquote}}{}
\IfFileExists{microtype.sty}{% use microtype if available
  \usepackage[]{microtype}
  \UseMicrotypeSet[protrusion]{basicmath} % disable protrusion for tt fonts
}{}
\makeatletter
\@ifundefined{KOMAClassName}{% if non-KOMA class
  \IfFileExists{parskip.sty}{%
    \usepackage{parskip}
  }{% else
    \setlength{\parindent}{0pt}
    \setlength{\parskip}{6pt plus 2pt minus 1pt}}
}{% if KOMA class
  \KOMAoptions{parskip=half}}
\makeatother
\usepackage{xcolor}
\IfFileExists{xurl.sty}{\usepackage{xurl}}{} % add URL line breaks if available
\IfFileExists{bookmark.sty}{\usepackage{bookmark}}{\usepackage{hyperref}}
\hypersetup{
  pdftitle={Interaction of age and gender on post-discharge quality-of-life in adult trauma patients in urban India -- a cohort study},
  hidelinks,
  pdfcreator={LaTeX via pandoc}}
\urlstyle{same} % disable monospaced font for URLs
\usepackage{longtable,booktabs,array}
\usepackage{calc} % for calculating minipage widths
% Correct order of tables after \paragraph or \subparagraph
\usepackage{etoolbox}
\makeatletter
\patchcmd\longtable{\par}{\if@noskipsec\mbox{}\fi\par}{}{}
\makeatother
% Allow footnotes in longtable head/foot
\IfFileExists{footnotehyper.sty}{\usepackage{footnotehyper}}{\usepackage{footnote}}
\makesavenoteenv{longtable}
\usepackage{graphicx}
\makeatletter
\def\maxwidth{\ifdim\Gin@nat@width>\linewidth\linewidth\else\Gin@nat@width\fi}
\def\maxheight{\ifdim\Gin@nat@height>\textheight\textheight\else\Gin@nat@height\fi}
\makeatother
% Scale images if necessary, so that they will not overflow the page
% margins by default, and it is still possible to overwrite the defaults
% using explicit options in \includegraphics[width, height, ...]{}
\setkeys{Gin}{width=\maxwidth,height=\maxheight,keepaspectratio}
% Set default figure placement to htbp
\makeatletter
\def\fps@figure{htbp}
\makeatother
\setlength{\emergencystretch}{3em} % prevent overfull lines
\providecommand{\tightlist}{%
  \setlength{\itemsep}{0pt}\setlength{\parskip}{0pt}}
\setcounter{secnumdepth}{-\maxdimen} % remove section numbering
\ifluatex
  \usepackage{selnolig}  % disable illegal ligatures
\fi

\title{Interaction of age and gender on post-discharge quality-of-life
in adult trauma patients in urban India -- a cohort study}
\author{}
\date{\vspace{-2.5em}}

\begin{document}
\maketitle

\hypertarget{introduction}{%
\section{Introduction}\label{introduction}}

Trauma contributes to one-tenth of the global disability-adjusted
life-years (DALYs), with low-and middle-income countries (LMICs) bearing
a disproportionate burden of the morbidity
{[}@GBD2019DemographicsCollaborators2020; @Haagsma2019{]}. To address
this burden it is important to understand the long-term outcomes of
trauma and the different factors associated with these outcomes,
especially in LMICs {[}@Kruithof2017; @Wisborg2017; @Rios-Diaz2017{]}.
This encompasses a range of socioeconomic outcomes including
health-related quality of life (QOL) {[}@Ahmed2017; @Nguyen2017;
@Hossain2020; @Jagnoor2019{]}.

Age and gender are associated with post-discharge QOL among trauma
patients. Elderly populations and women tend to have limited access to
resources, reduced social capital, disparities in support, poor
health-seeking behavior, and restricted education and employment
opportunities {[}@Amurwon2019;@Levasseur2015; @Brinda2016; @Gupta2019;
@Kennedy2020{]}. This can shape their post-discharge well-being and
outcomes after trauma {[}@Waqas2017; @Awang2018; @Quadir2019;
@Fabricius2020{]}.

Consequently, older age and being a women can make trauma patients more
vulnerable to poorer post-discharge QOL {[}@Gopinath2017;@Brown2017c;
@Mollayeva2018;@Rissanen2020{]}. There is some evidence that older women
may have higher morbidity in LMIC settings {[}@Carmel2019;
@Agrawal2014{]}, but there is little research on the interaction between
age and gender on QOL among post-discharge trauma patients in this
setting.

Understanding the interaction between age and gender on QOL may provide
insights for improving trauma management and developing support services
in LMIC settings {[}@Meara2015; @Babhulkar2019{]}. The aim of this study
is to assess the interaction of age and gender with post-discharge QOL
among adult trauma patients using the context of urban India.

\hypertarget{methods}{%
\section{Methods}\label{methods}}

\hypertarget{study-design}{%
\subsection{Study Design}\label{study-design}}

This study is a cross-sectional study using data from an ongoing
interrupted time series trial of trauma patients discharged from four
tertiary-care hospital in urban India between November 2019 and May
2021.

\hypertarget{setting}{%
\subsection{Setting}\label{setting}}

India accounts for nearly 20\% of global trauma burden
{[}@GBD2019DemographicsCollaborators2020{]}. More than one-tenth of all
the DALYs in India are due to trauma and it is among the top five causes
of morbidity {[}@Menon2019{]}. The patients were enrolled from the
on-going Trauma Audit Filters Trial (TAFT) in four participating
tertiary-care hospitals in Indian cities {[}@ClinicalTrials.gov2017{]}.
These were the Grant Medical College and Sir Jamshedjee Jeejeebhoy
Hospital in Mumbai, Lok Nayak Hospital of Maulana Azad Medical College
(MAMC) in Delhi, the Institute of Post-Graduate Medical Education and
Research and Seth Sukhlal Karnani Memorial Hospital (SSKM) in Kolkata
and St.~John's Medical College, Bengaluru. The first three are public
hospitals that have nominal fees catering to patients from lower
socioeconomic sections of the population, while the fourth is a
charitable private hospital catering to a mix of different socioeconomic
sections of the population.

\hypertarget{participants}{%
\subsection{Participants}\label{participants}}

We include patients aged 18, which is the legal age for consent in in
India {[}@IndianCouncilofMedicalResearch2006{]}, presenting to the
casualty department with a history of trauma---as per the V01-Y36,
chapter XX of the International Classification of Disease version 10
(ICD-10) {[}@WordHealthOrganization2016{]}--who are admitted and
discharged alive.

\hypertarget{variables}{%
\subsection{Variables}\label{variables}}

Age and gender were the main variables for this study. We also included
vital sign measures such as systolic blood pressure (SBP), respiratory
rate (RR), heart rate (HR), oxygen saturation (SPO), Injury severity
scores (ISS) and Glasgow Coma Scale (GCS) and injury etiology measures
like mode of transport to hospital, type of injury, mechanism of injury,
and length of hospital stay.

\hypertarget{outcomes}{%
\subsection{Outcomes}\label{outcomes}}

Health-related quality-of-life outcome was measured using the EQ-5D Tool
{[}@EuroQolGroup{]}. EQ-5D is a standardized measure of quality of life
using five dimensions: mobility, self-care, usual activities,
pain/discomfort, and anxiety/depression. Each dimension in the tool has
three levels: no problems, some problems, and extreme problems.
Additionally there is a visual analog scale (VAS) The patients were
followed up at 3-months. EQ-5D tool can be administered over the
telephone and has translations available in multiple Indian languages
{[}@EuroQolGroup{]}

Table 1: Description of study variables and outcomes

\begin{longtable}[]{@{}
  >{\raggedright\arraybackslash}p{(\columnwidth - 2\tabcolsep) * \real{0.32}}
  >{\raggedright\arraybackslash}p{(\columnwidth - 2\tabcolsep) * \real{0.65}}@{}}
\toprule
\endhead
Name & Description \\
Age & Patient's age rounded up to closest whole year \\
Sex & Patient's sex:

Female

Male \\
Vital Signs & Systolic blood pressure (SBP), respiratory rate (RR),
heart rate (HR), oxygen saturation (SPO), and Glasgow Coma Scale
(GCS) \\
Injury Etiology & Mode of transport, type of injury, mechanism of
injury \\
Injury Severity & Injury Severity Score \\
Outcomes & \\
Quality of Life & The participant's reply to the EQ-5D questionnaire and
VAS on health state at 3 months after arrival at the study site. \\
\bottomrule
\end{longtable}

\hypertarget{data-source}{%
\subsection{Data Source}\label{data-source}}

Data was collected by one dedicated independent project officer in each
of the hospitals who prospectively gather data on a standardized intake
form for eight hours per day, five days a week, by directly observing
the staff delivering trauma care. Vital signs such as systolic blood
pressure, heart-rate, and oxygen saturation was recorded by the project
officer independently. The project officer was rotated daily through
each eight-hour shift in the morning, evening and night. Data for the
variables was collected from patient records, or from the patient or
patient representatives when they are at the hospital. Additionally, the
project officer followed the patient or the patient relatives at
3-months after discharge by telephone the for information on the
outcomes and any missed variables.

\hypertarget{bias}{%
\subsection{Bias}\label{bias}}

There could be bias in collection of data and recording of vital signs
used to calculate injury severity. Adequate training of the project
officers, periodic quality control of the data with external project
officers and weekly online review meetings was done to reduce this bias.

\hypertarget{statistical-methods}{%
\subsection{Statistical Methods}\label{statistical-methods}}

Multivariable linear models was used to study the association of age and
gender with the QOL estimating 95\% confidence intervals and denote
associations with a p-value of less than 0.05 as statistically
significant {[}@Carey2013{]}. The statistical software R was used for
all statistical analyses {[}@RCoreTeam2015{]}.

\hypertarget{study-size}{%
\subsection{Study Size}\label{study-size}}

Simulation studies indicate that as many as 25 events, observations with
the outcome, and non-events or more per free parameter are required to
obtain stable estimates for logistic regression {[}@Courvoisier2011{]}.
Given that our logistic regression model included around 18 free
parameters, we need to include 375 events and at least as many
non-events. .

\hypertarget{ethics}{%
\subsection{Ethics}\label{ethics}}

Ethical clearance for the data collection was obtained from the four
participating hospitals as amendments to the existing ethical clearance
for the on-going TAFT project (Grant Medical College \& Sir J.J., Group
of Hospitals, Mumbai---No.~IEC/Pharm/CT/111/A/2017, Dated 22nd August
2017; Institute of Post-Graduate Medical Education, Kolkata---Memo
No.~IPGME\&R/IEC/2017/396, Dated 21st August 2017; Maulana Azad Medical
College, New Delhi-F.1/IEC/MAMC/(53/2/2016/No.97), Dated 3rd August
2016; St.~John's Medical College, Bengaluru---No.~IEC/1/671/2017, Dated
24th August, 2017). Waiver of informed consent was granted for
collection of clinical data which was routinely collected for the
patients, as they were all admitted after trauma, often arriving in an
altered level of consciousness and in severe physical and psychological
distress. The amendment granted permission to collect the additional
data necessary for this study (Grant Medical College \& Sir J.J., Group
of Hospitals, Mumbai---No.~IEC/Pharm/CT/2059/2019, Dated 16th September
2019; St.~John's Medical College, Bengaluru---No.~IEC/1/530/2019, Dated
25th June, 2019).

\hypertarget{data-management}{%
\subsection{Data management}\label{data-management}}

Each center was assigned a center identification number and each patient
a locally unique study identification number. Project officers first
entered data on paper without any personal identification data. The
project officers then transferred this data to an electronic format
using a dedicated data entry application. The electronic data did not
include any direct identifiers such as name, hospital record number, and
telephone numbers. The only way to link an electronic record to a paper
intake form was by combining the record's hospital and study
identification numbers. Paper forms were kept locally at each center for
the duration required by locally applicable laws and regulations, or at
least five years, whichever is longest. The adequacy of their storage
was the responsibility of the principal investigator at each center.
Care was taken taken to ensure that at no time where they stored with
less than reasonable care.

\hypertarget{results}{%
\section{Results}\label{results}}

\begin{longtable}[]{@{}llll@{}}
\caption{Sample characteristics}\tabularnewline
\toprule
& 5 & 14 & 42 \\
\midrule
\endfirsthead
\toprule
& 5 & 14 & 42 \\
\midrule
\endhead
3 & Age in years (median {[}IQR{]}) & & 36.0 {[}26.0, 50.0{]} \\
4 & Sex (\%) & Female & 482 (19.9) \\
5 & & Male & 1937 (80.1) \\
6 & Mechanism of injury (\%) & Animal bites & 16 (0.7) \\
7 & & Assault & 149 (6.1) \\
8 & & Falls & 563 (23.2) \\
9 & & Other & 287 (11.8) \\
10 & & Railway injuries & 43 (1.8) \\
11 & & Road traffic injuries & 1369 (56.4) \\
12 & Type of injury (\%) & Blunt & 2372 (98.1) \\
13 & & Penetrating & 45 (1.9) \\
14 & Mode of transport (\%) & Ambulance & 1670 (69.1) \\
15 & & Police van & 119 (4.9) \\
16 & & Private Vehicles & 620 (25.6) \\
17 & & On Foot & 9 (0.4) \\
18 & Transferred (\%) & Direct Admissions & 527 (21.8) \\
19 & & Transferred & 1890 (78.2) \\
20 & SBP (median {[}IQR{]}) & & 118.0 {[}110.0, 130.0{]} \\
21 & RR (median {[}IQR{]}) & & 21.0 {[}18.0, 22.0{]} \\
22 & HR (median {[}IQR{]}) & & 84.0 {[}78.0, 93.0{]} \\
23 & SpO2 (median {[}IQR{]}) & & 98.0 {[}97.0, 98.0{]} \\
24 & GCS (median {[}IQR{]}) & & 15.0 {[}12.0, 15.0{]} \\
25 & EQ5D Health Status (median {[}IQR{]}) & & 80.0 {[}65.0, 90.0{]} \\
26 & EQ5D Mobility (\%) & No Problems & 360 (54.8) \\
27 & & Some Problems & 216 (32.9) \\
28 & & Confined to bed & 81 (12.3) \\
29 & EQ5D Self Care (\%) & No Problems & 415 (63.2) \\
30 & & Some Problems & 182 (27.7) \\
31 & & Unable to wash or dress & 60 (9.1) \\
32 & EQ5D Usual Activities (\%) & No Problems & 283 (43.0) \\
33 & & Some Problems & 245 (37.2) \\
34 & & Unable to perform usual activities & 130 (19.8) \\
35 & EQ5D Pain/Discomfort (\%) & No Pain & 228 (34.7) \\
36 & & Moderate Pain & 401 (60.9) \\
37 & & Extreme Pain & 29 (4.4) \\
38 & EQ5D Anxiety/Depression (\%) & No Anxious/depressed & 385 (58.9) \\
39 & & Moderately Anxious/depressed & 200 (30.6) \\
40 & & Extremely Anxious/depressed & 69 (10.6) \\
41 & 30 day mortality (\%) & Alive & 1196 (74.7) \\
42 & & Dead & 406 (25.3) \\
\bottomrule
\end{longtable}

Out of a total of 2427 trauma patients enrolled, 1796 were excluded for
missing data. The final cohort of 631 included in this study had a
median age of 36 years (IQR: 26-50) with 80.07 per cent of patients were
male. The most common mechanisms of injury were Road traffic injuries (
per cent). Majority of the injuries were blunt (98.14 per cent) and 78.2
per cent were transferred to the study hospitals. The all cause 30-day
mortality was 16.73 per cent. The median severity score was x. Details
of the study population is given in Table 2.

\hypertarget{eq-5d-scores}{%
\subsection{EQ-5D Scores}\label{eq-5d-scores}}

652 completed the EQ-5D questionnaire at 3 months after discharge. The
mean EQ5D index score was 76 (SD = 20.5). Three months after discharge,
young males (18-32 years) had the highest score while middle-aged
females reported the lowest score. Just over half the patients (54.8\%)
reported no problems with mobility while two-thirds reported no problems
with self-care. While less than half the patients (43.0\%) could carry
on usual activities without any problems and only one-third of the
patients (34.7\%) reported no pain or discomfort after three-months of
post-discharge. Around 40\% of the patients reported experiencing some
form of anxiety or depression. Again, the proportion of young males
reporting any problems across all the five domains was the lowest while
the proportion middle-aged females reported experiencing the problems
across all the five domains was the highest. The overall EQ-5D scores
are provided in Table 3.

\hypertarget{eq-5d-in-relation-to-gender-and-age}{%
\subsection{EQ-5D in relation to gender and
age}\label{eq-5d-in-relation-to-gender-and-age}}

Overall females had a slightly higher mean EQ5D index score (77.6027397)
than males (76.5790514). On running a linear regression of index scores
across sex across age after adjusting for injury etiology, vitals, and
severity, young males aged (aged 18-32) performed better than
middle-aged (aged 33-49) and old (aged 50 and above) males. On the other
hand, old females had higher scores than young and middle aged females.
Middle-age group had the lowest performance, with middle-aged faring the
worst (Fig 1).

Regression analysis in the mobility dimension keeping young males as the
reference group, shows that the adjusted odds of reporting mobility
problems was 1.0557517 in young females (aged 18-32), 1.0400443
middle-aged females (aged 33-49), and 1.0430457 old females (aged 50 and
above). In middle-aged males (aged 33-49) the odds were 1.4203472 and in
old males (aged 50 and above) the odds were 1.0578552. Thus, all groups
had higher odds of reporting some mobility problems than young males.

Assessing the regression analysis in the self-care dimension, showed the
adjusted odds of reporting problems performing self-care tasks was
1.3238941 in young females, 1.1692589 in middle-aged females, and
0.7880519 old females compared to young males. On the other hand, the
odds of reporting these problems in middle-aged males was 1.3303163 and
in old males it was 1.0728229times than young males. Only old females
had lower odds of reporting problems with self-care than young males.

In the dimension of being able to perform usual activities, regression
analysis indicated that the adjusted odds of reporting any problems in
performing was 1.3246203 in young females, 0.798196 in middle aged
females, and 0.8050678 old females with respect to young males. The odds
of reporting these problems were 1.6970086 in middle aged males and
1.2401893 times in in old males. Thus, middle-aged and old females had
lower odds of having problems performing usual activities than young
males.

The adjusted odds of reporting any form of pain and discomfort were
1.7825876 in young females, 0.5814462 in middle-aged females, and
0.6054132 old females compared to young males. Among middle aged and old
males these odds were 2.1505875 and 1.9242736 respectively. Thus, middle
aged and old females had lower odds of reporting pain than young males.

In the anxiety and depression dimension, keeping young males as the
reference group, the adjusted odds of reporting any form of anxiety or
depression were 1.3986886 in young females, 0.4864969 middle-aged
females, and 1.1036321 old females. In middle-aged males the odds was
1.5951319 and in old males the odds was 0.8453032. So, younger and
middle-aged females had lower odds of reporting having any form of
anxiety and depression than young males.

Though most of the odds of reporting any problems across the EQ5D
domains were not statistically significant, middle aged males reported
significant problems with pain and usual activity while old males
reported significant problems with pain and anxiety (Table 4). The
interaction of age and gender on the index EQ5D score and across its 5
domains was found to be not statistically significant.

\hypertarget{discussion}{%
\section{Discussion}\label{discussion}}

This study examined the interaction of age and gender on QOL in trauma
patients discharged from four tertiary-care hospital in urban India at
3-months after discharge. We did not find statistically significant
differences across all age groups and gender as well as no interaction
effect between age and gender in this population. The sub-group analysis
showed differences in QOL between these groups. In each of the five EQ5D
domains, odds of reporting problems varied between the groups. We found
middle aged men report significantly lower QOL scores and was due to
having higher odds of reporting problems in mobility, anxiety or
depression, and especially pain, and usual activities.

The mean EQ5D index score of 76 of trauma patients in this study is
lower than the general population norm for South Asia of 85
{[}@Kularatna2014{]}. This indicates that even after 3-months since
discharge, trauma patients in urban India had not yet reached the
pre-injury norm of QOL for the region. More services to address their
needs.

In this study middle-aged cohort, especially women, seem to be having
the odds for the poorest QOL outcomes.

Pain and discomfort was the domain that reported the highest proportion
of reporting problems followed by usual activities.

However, middle-aged and older women had the lowest odds of reporting
problems in pain and usual activities than other groups. Could be due to
the social structure where men and younger women going to formal
employment, which is deemed in higher value than domestic chores, that
women engage in. Pain reporting tends to be lower in women again due
social factors. Similarly women reported lowest odds of anxiety and
depression again could be due to social factors

Young men had highest odds of reporting problems with mobility than
other groups could be due to the group having more access to move/travel
than other age groups.

\textless!--Just wondering why having a higher proportion of those
reporting any problem does not translate to higher odds of having that
problem. For e.g.Middle-aged women report the highest proportion of any
pain but have the lowest odds of reporting any pain than young
men.--!\textgreater{}

\hypertarget{conclusion}{%
\section{Conclusion}\label{conclusion}}

\hypertarget{references}{%
\section{References}\label{references}}

\end{document}
